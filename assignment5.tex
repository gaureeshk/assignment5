\documentclass[journal,12pt,twocolumn]{IEEEtran}
\usepackage{graphicx}
\usepackage{paralist}
\usepackage{setspace}
\usepackage{gensymb}
\singlespacing
\usepackage[cmex10]{amsmath}
\usepackage{amsthm}
\usepackage{amsmath}
\usepackage{amssymb}
\usepackage{mathrsfs}
\usepackage{txfonts}
\usepackage{stfloats}
\usepackage{bm}
\usepackage{cite}
\usepackage{cases}
\usepackage{subfig}
\usepackage{longtable}
\usepackage{multirow}

\usepackage{enumitem}
\usepackage{mathtools}
\usepackage{steinmetz}
\usepackage{tikz}
\usepackage{circuitikz}
\usepackage{verbatim}
\usepackage{tfrupee}
\usepackage[breaklinks=true]{hyperref}
\usepackage{graphicx}
\usepackage{tkz-euclide}

\usetikzlibrary{calc,math}
\usepackage{listings}
    \usepackage{color}                                            %%
    \usepackage{array}                                            %%
    \usepackage{longtable}                                        %%
    \usepackage{calc}                                             %%
    \usepackage{multirow}                                         %%
    \usepackage{hhline}                                           %%
    \usepackage{ifthen}                                           %%
    \usepackage{lscape}     
\usepackage{multicol}
\usepackage{chngcntr}

\DeclareMathOperator*{\Res}{Res}

\renewcommand\thesection{\arabic{section}}
\renewcommand\thesubsection{\thesection.\arabic{subsection}}
\renewcommand\thesubsubsection{\thesubsection.\arabic{subsubsection}}

\renewcommand\thesectiondis{\arabic{section}}
\renewcommand\thesubsectiondis{\thesectiondis.\arabic{subsection}}
\renewcommand\thesubsubsectiondis{\thesubsectiondis.\arabic{subsubsection}}


\hyphenation{op-tical net-works semi-conduc-tor}
\def\inputGnumericTable{}                                 %%

\lstset{
%language=C,
frame=single, 
breaklines=true,
columns=fullflexible
}
\begin{document}


\newtheorem{theorem}{Theorem}[section]
\newtheorem{problem}{Problem}
\newtheorem{proposition}{Proposition}[section]
\newtheorem{lemma}{Lemma}[section]
\newtheorem{corollary}[theorem]{Corollary}
\newtheorem{example}{Example}[section]
\newtheorem{definition}[problem]{Definition}

\newcommand{\BEQA}{\begin{eqnarray}}
\newcommand{\EEQA}{\end{eqnarray}}
\newcommand{\define}{\stackrel{\triangle}{=}}
\bibliographystyle{IEEEtran}
\raggedbottom
\setlength{\parindent}{0pt}
\providecommand{\mbf}{\mathbf}
\providecommand{\pr}[1]{\ensuremath{\Pr\left(#1\right)}}
\providecommand{\qfunc}[1]{\ensuremath{Q\left(#1\right)}}
\providecommand{\sbrak}[1]{\ensuremath{{}\left[#1\right]}}
\providecommand{\lsbrak}[1]{\ensuremath{{}\left[#1\right.}}
\providecommand{\rsbrak}[1]{\ensuremath{{}\left.#1\right]}}
\providecommand{\brak}[1]{\ensuremath{\left(#1\right)}}
\providecommand{\lbrak}[1]{\ensuremath{\left(#1\right.}}
\providecommand{\rbrak}[1]{\ensuremath{\left.#1\right)}}
\providecommand{\cbrak}[1]{\ensuremath{\left\{#1\right\}}}
\providecommand{\lcbrak}[1]{\ensuremath{\left\{#1\right.}}
\providecommand{\rcbrak}[1]{\ensuremath{\left.#1\right\}}}
\theoremstyle{remark}
\newtheorem{rem}{Remark}
\newcommand{\sgn}{\mathop{\mathrm{sgn}}}
\providecommand{\abs}[1]{\left\vert#1\right\vert}
\providecommand{\res}[1]{\Res\displaylimits_{#1}} 
\providecommand{\norm}[1]{\left\lVert#1\right\rVert}
%\providecommand{\norm}[1]{\lVert#1\rVert}
\providecommand{\mtx}[1]{\mathbf{#1}}
\providecommand{\mean}[1]{E\left[ #1 \right]}
\providecommand{\fourier}{\overset{\mathcal{F}}{ \rightleftharpoons}}
%\providecommand{\hilbert}{\overset{\mathcal{H}}{ \rightleftharpoons}}
\providecommand{\system}{\overset{\mathcal{H}}{ \longleftrightarrow}}
	%\newcommand{\solution}[2]{\textbf{Solution:}{#1}}
\newcommand{\comb}[2]{{}^{#1}\mathrm{C}_{#2}}
\newcommand{\solution}{\noindent \textbf{Solution: }}
\newcommand{\cosec}{\,\text{cosec}\,}
\newcommand{\cosec}{}
\providecommand{\dec}[2]{\ensuremath{\overset{#1}{\underset{#2}{\gtrless}}}}
\newcommand{\myvec}[1]{\ensuremath{\begin{pmatrix}#1\end{pmatrix}}}
\newcommand{\mydet}[1]{\ensuremath{\begin{vmatrix}#1\end{vmatrix}}}
\numberwithin{equation}{subsection}
\makeatletter
\@addtoreset{figure}{problem}
\makeatother
\let\StandardTheFigure\thefigure
\let\vec\mathbf
\renewcommand{\thefigure}{\theproblem}
\def\putbox#1#2#3{\makebox[0in][l]{\makebox[#1][l]{}\raisebox{\baselineskip}[0in][0in]{\raisebox{#2}[0in][0in]{#3}}}}
     \def\rightbox#1{\makebox[0in][r]{#1}}  
     \def\centbox#1{\makebox[0in]{#1}}
     \def\topbox#1{\raisebox{-\baselineskip}[0in][0in]{#1}}
     \def\midbox#1{\raisebox{-0.5\baselineskip}[0in][0in]{#1}}
\vspace{3cm}
\title{Assignment 5}
\author{Gaureesha Kajampady - EP20BTECH11005}
\maketitle  
\newpage
\bigskip
\renewcommand{\thefigure}{\theenumi}
\renewcommand{\thetable}{\theenumi}
Download latex-tikz codes from 
%
\begin{lstlisting}
https://github.com/gaureeshk/assignment5/blob/main/assignment5.tex
\end{lstlisting}
\section{Problem}
(CSIR UGC NET EXAM (Dec 2014), Q.105)\\
Let $X_{1},X_{2},X_{3},...$ be independent random variables with $E(X_{k})=0$ and Var($X_{k}$)=k. Let $S_{n}\sum_}_{k=1}^{n}X_{k}$.Then as n $\to \infty$,\\
\begin{enumerate}
    \item{$\frac{S_{n}}{n^{\frac{3}{2}}} \to $0 in probability}\\
    \item{$\frac{S_{n}}{n^{\frac{3}{2}}} \to $0 in distribution}\\
    \item{$\frac{S_{n}X_{n}}{n^{\frac{5}{2}}} \to $0 in distribution}\\
    \item{$\frac{S_{n}X_{n}}{n^{\frac{5}{2}}} \to $0 in probability}\\
\end{enumerate}
\section{Solution} 
\begin{definition}
    (convergence in probability)\\
    Let $X_1, X_2$, . . . be an infinite sequence of random variables, and let $Y$ be another random variable. Then the sequence $\{X_{n}\}$ converges in probability to $Y$, if for all $\epsilon >0$, 
    \begin{align}
        \lim_{n \to \infty}\pr{|X_n-Y|\geq \epsilon}=0
    \end{align}
And we write as $n \to \infty$, $X_n \to Y$ in probability.
\end{definition}
\begin{definition}
        (Convergence in Distribution)\\
        Let $X, X_{1}, X_{2},\dots $ be random variables. Then we say that the sequence $\{X_n\}$ converges to X, if $\forall x \in R^1$ such that $\pr{X=x}=0$, we have
        \begin{align}
            \lim_{n \to \infty}\pr{X_n \leq x}=\pr{X \leq x}.
        \end{align}
        \end{definition}
\begin{theorem} \label{1}
        If $X_n \to X$ in probability, $X_n \to X$ in distribution.
        \end{theorem}
\begin{theorem} \label{2}
(Chebyshev's inequality)\\
        Let X be a random variable with finite expected value $\mu$ and finite non-zero variance $\sigma^2$. Then for any real number k > 0,
        \begin{align}
            \pr{\abs{X-\mu}\geq \epsilon} \leq \frac{\sigma^{2}}{\epsilon^{2}}
        \end{align}
\end{theorem}
We know that,\\Since $S_{n}$ is a sum of random variables, it itself is a random variable.
We also know that, 
\begin{align} 
E\brak{\sum_{i=1}^{n}X_{i}}=\sum_{i=1}^{n}E(X_{i}) 
\end{align} 
Hence, 
\begin{align} 
E\brak{S_{n}}=\sum_{i=0}^{n}0=0 \\
\implies E\brak{\frac{S_{n}}{n^{\frac{3}{2}}}}=0
\end{align} 
Since the random variables $X_{1},X_{2}$,.. are independent \begin{align} 
Var(\sum_{i=1}^{n}X_{i})=\sum_{i=1}^{n}Var(X_{i})\\
Var(S_{n})=\sum_{i=1}^{n}i=\frac{n \times (n+1)}{2}\\
\implies Var\brak{\frac{S_{n}}{n^{\frac{3}{2}}}}&=\frac{Var(S_{n})}{n^{3}}\\
&=\frac{(n+1)}{2n^{2}}
\end{align} 
Hence as n $\to \infty$,
\begin{align}
    E\brak{\frac{S_{n}}{n^{\frac{3}{2}}}} \to 0\\
    Var\brak{\frac{S_{n}}{n^{\frac{3}{2}}}} \to 0\\
\end{align}
Using theorem \ref{2} (Chebyshev's inequality),
\begin{align}
    \pr{\abs{\frac{S_{n}}{n^{\frac{3}{2}}}-\mu}\geq \epsilon} \leq \frac{\sigma^{2}}{\epsilon^{2}}
\end{align}
As n$\to \infty$, $\mu \to 0$ and $\sigma \to 0$
\begin{align}
    \implies \frac{\sigma^{2}}{\epsilon^{2}} \to 0\\
    \implies  \pr{\abs{\frac{S_{n}}{n^{\frac{3}{2}}}-0}\geq \epsilon} \to 0
\end{align}
Hence $\frac{S_{n}}{n^{\frac{3}{2}}}$ converges to 0 in probability.\\
By using theorem \ref{1},\\
$\implies$ $\frac{S_{n}}{n^{\frac{3}{2}}}$ converges to 0 in distribution.
\end{document}





















